% \iffalse meta-comment
%
% Copyright (C) 2024 by Lukas Heindl <<+email+>>
% ---------------------------------------------------------------------------
% This work may be distributed and/or modified under the
% conditions of the LaTeX Project Public License, either version 1.3
% of this license or (at your option) any later version.
% The latest version of this license is in
%   http://www.latex-project.org/lppl.txt
% and version 1.3 or later is part of all distributions of LaTeX
% version 2005/12/01 or later.
%
% This work has the LPPL maintenance status `maintained'.
%
% The Current Maintainer of this work is Lukas Heindl.
%
% This work consists of the files beamercolorthemeCatppuccin.dtx and beamercolorthemeCatppuccin.ins
% and the derived filebase beamercolorthemeCatppuccin.sty.
%
% \fi
%
% \iffalse
%<*driver>
\ProvidesFile{beamercolorthemeCatppuccin.dtx}
%</driver>
%<package>\NeedsTeXFormat{LaTeX2e}[1999/12/01]
%<package>\ProvidesPackage{beamercolorthemeCatppuccin}
%<*package>
    [2024/02/25 v0.0.1 catppuccin beamer color theme]
%</package>
%
%<*driver>
\documentclass{ltxdoc}
\usepackage{beamerarticle}
\usepackage{beamercolorthemeCatppuccin}[2024/02/25]
\usepackage{tabularx, booktabs}
\EnableCrossrefs
\CodelineIndex
\RecordChanges
\begin{document}
  \DocInput{beamercolorthemeCatppuccin.dtx}
  \PrintChanges
  \PrintIndex
\end{document}
%</driver>
% \fi
%
% \CheckSum{68}
%
% \CharacterTable
%  {Upper-case    \A\B\C\D\E\F\G\H\I\J\K\L\M\N\O\P\Q\R\S\T\U\V\W\X\Y\Z
%   Lower-case    \a\b\c\d\e\f\g\h\i\j\k\l\m\n\o\p\q\r\s\t\u\v\w\x\y\z
%   Digits        \0\1\2\3\4\5\6\7\8\9
%   Exclamation   \!     Double quote  \"     Hash (number) \#
%   Dollar        \$     Percent       \%     Ampersand     \&
%   Acute accent  \'     Left paren    \(     Right paren   \)
%   Asterisk      \*     Plus          \+     Comma         \,
%   Minus         \-     Point         \.     Solidus       \/
%   Colon         \:     Semicolon     \;     Less than     \<
%   Equals        \=     Greater than  \>     Question mark \?
%   Commercial at \@     Left bracket  \[     Backslash     \\
%   Right bracket \]     Circumflex    \^     Underscore    \_
%   Grave accent  \`     Left brace    \{     Vertical bar  \|
%   Right brace   \}     Tilde         \~}
%
%
% \changes{v0.0.1}{2024/02/25}{Converted to DTX file}
%
% \DoNotIndex{\newcommand,\newenvironment}
%
% \providecommand*{\url}{\texttt}
% \GetFileInfo{beamercolorthemeCatppuccin.dtx}
% \title{The \textsf{beamercolorthemeCatppuccin} package}
% \author{Lukas Heindl \\ \url{<+email+>} \\ \url{https://github.com/atticus-sullivan/beamercolorthemeCatppuccin}}
% \date{\fileversion~from \filedate}
%
% \maketitle
% \begin{abstract}
% The LaTeX package \pkg{beamercolorthemeCatppuccinPalette} brings the
% catppuccin theme to beamer.
% \end{abstract}
%
% \section{Introduction}
%
% Catppuccin\footnote{\url{https://github.com/catppuccin/catppuccin}} is a nice
% pastel theme in 4 flavours/styles (\emph{Latte}, \emph{Frapp\'e},
% \emph{Macchiato}, \emph{Mocha}). This package makes use of the
% \verb|catppuccinPalette|\footnote{\url{TODO}} package to bring these color
% themes to beamer.
%
% \section{Usage}
% Just load the color theme with \verb|\usecolortheme{Catppuccin}| and use
% beamer just like usually.
%
% Note that this package also defines that \verb|\emph| uses the (fg) beamer
% color \verb|structure|.
%
% As this package loads the \verb|catppuccinPalette| package, we provide the
% \verb|paletteOpts| package option which is passed to that package (to avoid
% option clashes).

% This package also allows you to set a specific style which sets style of the
% color scheme being used via the package option \verb|style|.
%
% \subsection{Package options}
% \begin{tabularx}{\linewidth}{lX}
%   \emph{Option} & \emph{Description}
%   \\\toprule
%   \verb|style| &
%   set the style of the color scheme being used for beamer. Possible values:
%   \verb|Latte| (default+initial), \verb|latte|, \verb|Frappe|, \verb|frappe|,
%   \verb|Mocha|, \verb|mocha|
%   \\
%   \verb|paletteOpts| &
%   pass options to the catppuccinPalette package when loading it
%   \\
%   \\\bottomrule
% \end{tabularx}
%
% \StopEventually{}
%
% \section{Implementation}
%
% \iffalse
%<*package>
% \fi
%
%    \begin{macrocode}
\ProvidesPackage{beamercolorthemeCatppuccin}
  [2024/02/06 v0.0.1 A beamer theme using Catppuccin color scheme]
\@ifclassloaded{beamer}{
\mode<presentation>
\DeclareOptionBeamer{style}{\def\beamer@catppuccintheme@style{#1}}
\DeclareOptionBeamer{paletteOpts}{\def\beamer@catppuccintheme@paletteOpts{#1}}
\ExecuteOptionsBeamer{style=Latte} % set default
\ProcessOptionsBeamer
%
\RequirePackage[\beamer@catppuccintheme@paletteOpts]{catppuccinPalette}
%    \end{macrocode}
% \setbeamercolor*{structure}{
%   bg=Cat\beamer@catppuccintheme@style Red!20,
%   fg=Cat\beamer@catppuccintheme@style Red
%   }
% \setbeamercolor*{structure}{
%   bg=Cat\beamer@catppuccintheme@style Peach!20,
%   fg=Cat\beamer@catppuccintheme@style Peach
%   }
%    \begin{macrocode}
\setbeamercolor*{structure}{
  bg=Cat\beamer@catppuccintheme@style Green!20,
  fg=Cat\beamer@catppuccintheme@style Green
}
%    \end{macrocode}
% \setbeamercolor*{structure}{
%   bg=Cat\beamer@catppuccintheme@style Teal!20,
%   fg=Cat\beamer@catppuccintheme@style Teal
%   }
% \setbeamercolor*{structure}{
%   bg=Cat\beamer@catppuccintheme@style Sapphire!20,fg=Cat\beamer@catppuccintheme@style Sapphire}
% \setbeamercolor*{structure}{
%   bg=Cat\beamer@catppuccintheme@style Lavender!20,
%   fg=Cat\beamer@catppuccintheme@style Lavender
%   }
%
%    \begin{macrocode}
\setbeamercolor*{normal text}{
  bg=Cat\beamer@catppuccintheme@style Base,
  fg=Cat\beamer@catppuccintheme@style Text
}
\setbeamercolor*{section in toc}{
  fg=Cat\beamer@catppuccintheme@style Text,
  bg=Cat\beamer@catppuccintheme@style Base
}
\setbeamercolor*{alerted text}{
  use=structure,fg=structure.fg!80!gray}
\setbeamercolor*{palette primary}{
  use=structure,
  fg=structure.fg!60!Cat\beamer@catppuccintheme@style Text,
  bg=gray!30!Cat\beamer@catppuccintheme@style Base
}
\setbeamercolor*{palette secondary}{
  use=structure,
  fg=structure.fg!70!Cat\beamer@catppuccintheme@style Text,
  bg=gray!15!Cat\beamer@catppuccintheme@style Base
}
\setbeamercolor*{palette tertiary}{
  use=structure,
  bg=structure.fg!80!Cat\beamer@catppuccintheme@style Text,
  fg=gray!10!Cat\beamer@catppuccintheme@style Base
}
\setbeamercolor*{palette quaternary}{
  use=structure,
  fg=structure.fg,
  bg=gray!5!Cat\beamer@catppuccintheme@style Base
}
\setbeamercolor*{sidebar}{
  use=structure,
  fg=structure.fg,
  bg=gray!15!Cat\beamer@catppuccintheme@style Base
}
\setbeamercolor*{palette sidebar primary}{
  use=structure,
  fg=structure.fg!10!Cat\beamer@catppuccintheme@style Text
}
\setbeamercolor*{palette sidebar secondary}{
  fg=Cat\beamer@catppuccintheme@style Base
}
\setbeamercolor*{palette sidebar tertiary}{
  use=structure,
  fg=structure.fg!50!Cat\beamer@catppuccintheme@style Text
}
\setbeamercolor*{palette sidebar quaternary}{
  fg=gray!10!Cat\beamer@catppuccintheme@style Base
}
\setbeamercolor*{titlelike}{
  parent=structure,
  bg=Cat\beamer@catppuccintheme@style Base
}
\setbeamercolor*{frametitle}{
  bg=gray!10!Cat\beamer@catppuccintheme@style Base
}
\setbeamercolor*{frametitle right}{
  bg=gray!60!Cat\beamer@catppuccintheme@style Base
}
\setbeamercolor*{separation line}{}
\setbeamercolor*{fine separation line}{}
%    \end{macrocode}
% from orchid
%    \begin{macrocode}
\setbeamercolor{block title}{
  use=structure,
  fg=Cat\beamer@catppuccintheme@style Base,
  bg=structure.fg!75!Cat\beamer@catppuccintheme@style Text
}
\setbeamercolor{block title alerted}{
  use=alerted text,
  fg=Cat\beamer@catppuccintheme@style Base,
  bg=alerted text.fg!75!Cat\beamer@catppuccintheme@style Text
}
\setbeamercolor{block title example}{
  use=example text,
  fg=Cat\beamer@catppuccintheme@style Base,
  bg=example text.fg!75!Cat\beamer@catppuccintheme@style Text
}
\setbeamercolor{block body}{
  parent=normal text,
  use=block title,
  bg=block title.bg!10!bg
}
\setbeamercolor{block body alerted}{
  parent=normal text,
  use=block title alerted,
  bg=block title alerted.bg!10!bg
}
\setbeamercolor{block body example}{
  parent=normal text,
  use=block title example,
  bg=block title example.bg!10!bg
%    \end{macrocode}

%    \begin{macrocode}
}
%    \end{macrocode}
%
% \begin{macro}{\emph}
% Redefine emph to use the color defined in beamer
%    \begin{macrocode}
\renewcommand{\emph}[1]{{\usebeamercolor[fg]{structure}#1}}
%    \end{macrocode}
% \end{macro}
% close if beamer loaded
%    \begin{macrocode}
\mode<all>
}
%    \end{macrocode}

%
% \iffalse
%</package>
% \fi
%
% \Finale
\endinput
